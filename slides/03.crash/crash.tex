\documentclass{beamer}
\usepackage{amsmath}
\usepackage{rotating}
\usepackage{graphicx}
\usepackage{multimedia}
\usepackage{listings}

\useinnertheme[shadow=true]{rounded}
\useoutertheme{shadow}
\usecolortheme{orchid}
\usecolortheme{whale}

\mode<presentation>

\newcommand{\dif}{\, \mathrm{d}}
\newcommand{\diff}[2]{\frac{\mathrm{d}#1}{\mathrm{d}#2}}
\newcommand{\partdiff}[2]{\frac{\partial #1}{\partial #2}}

\title{TMA4280 - Introduction to supercomputing}
\subtitle{Bash, compilers, make, cmake, scripting}
\author{Arne Morten Kvarving}
\institute{NTNU and SINTEF ICT}
\date{January 2014}

\begin{document}

\maketitle
\begin{frame}\frametitle{The shell - your best friend}
  \begin{itemize}
    \item Computers are commonly operated using graphical user interfaces.
    \item Supercomputers are not common computers.
    \item Some systems have web interfaces for dispatching jobs - ``cloud computing''.
    \item These are only frontends to a shell based system.
    \item Almost all (probably 99\%) of these systems are based on some Unix (mostly Linux today).
    \item Many different shells exists - csh, tcsh, sh,  $\cdots$.
    \item The most common shell on Linux is the Bourne Again SHell - \emph{bash}.
  \end{itemize}
\end{frame}

\begin{frame}\frametitle{Bash}
  \begin{itemize}
    \item The (bash) shell is very powerful - built-in scripting.
    \item Scripts used for jobs on the supercomputer facilities.
    \item Basic shell usage will be covered.
    \item Some basic utilities available on the shell will be covered.
    \item Compiling software on the shell will be covered, by hand as well as automated.
    \item Some basics of shell scripting will be covered.
  \end{itemize}
\end{frame}

\begin{frame}\frametitle{Bash - basic navigation}
  \begin{itemize}
    \item[ls] - LiSt \\
      This is used to list the files in a directory.
    \item[cd] - Change Directory \\
      This is used to navigate around in the directory structure - e.g. 'cd foo'.
    \item[mkdir] - MaKe DIRectory \\
      This is used to create a new directory.
    \item[rm] - ReMove \\
      This is used to delete a file or directory.
    \item[mv] - MoVe \\
      This is used to move (rename) a file or directory.
  \end{itemize}
\end{frame}

\begin{frame}\frametitle{Bash - long live lazyness}
  \begin{itemize}
    \item All commands have readily available documentation.
    \item \emph{man foo} or \emph{info foo} documents parameters and explains usage scenarios.
    \item Bash has \emph{tab completion} - whenever a command/file/directory is qualified hit tab to 
          have bash complete it to lessen the amount of typing required.
    \item Use the \emph{arrow keys} to recall previously executed commands.
    \item Use \emph{ctrl-r} to search in previously executed commands (anywhere, not just the start).
  \end{itemize}
\end{frame}

\begin{frame}\frametitle{Bash - handy utilities}
  \begin{itemize}
    \item[cat] This is used to show the contents of a file. \\
      cat test.c
    \item[less] This is used to show the contents of a file with pagenation. \\
      less test.c
    \item[find] You can use the \emph{find} command to locate a lost file. \\
      find . -name "*.c"
    \item[grep] You can use the \emph{grep} command to search for text in a file. \\
      grep "foo" test.c
    \item[sed] You can use the \emph{sed} command to search and replace text in a file.
      sed -e 's/foo/bar/g' -i test.c
  \end{itemize}
\end{frame}

\begin{frame}\frametitle{Bash - output redirection}
  \begin{itemize}
    \item Sometimes we want to combine several utilities, in the sense that the output from
          one command is the input to another one.
    \item This can be achieved through \emph{pipes}. \\
          cat test.c $\vert$ grep "foo" $\vert$ awk - F ' ' '$\{print\ \$2\}$'
    \item Sometimes we want to redirect the output of a program to a file.
    \item This can be achieved through $>$. \\
        grep "foo" test.c $>$ output.txt
    \item You can append to the output file through $>>$. \\
        grep "bar" test.c $>>$ output.txt
  \end{itemize}
\end{frame}

\begin{frame}\frametitle{Bash - compiling software}
  \begin{itemize}
    \item Software is compiled by invoking a compiler, and finally a linker, on the shell.
    \item Compiler commands vary across systems.
    \item Compiler commands depend on the language used.
    \item However, there are a lot of similarities as well.
    \item We here consider Linux systems using the GNU compilers.
    \item The basic approach is:
      \begin{itemize}
        \item Compile sources into object (.o) files.
        \item Link objects into a library (.a/.so) or an executable.
      \end{itemize}
  \end{itemize}
\end{frame}

\begin{frame}\frametitle{Our first C program}
  \begin{itemize}
    \item We first consider compiling a simple hello world program, sources given below.
      \lstinputlisting[language=C]{../../examples/crash/hello.c}
  \end{itemize}
\end{frame}

\begin{frame}\frametitle{Our first C program contd.}
  \begin{itemize}
    \item We only have a single source file.
    \item We can then do the compilation and linking in a single command. \\
      gcc -o hello hello.c
    \item We might want to turn on compiler optimizations. \\
      gcc -o hello hello.c -O2
    \item We might want to include debugging info. \\
      gcc -o hello hello.c -g
  \end{itemize}
\end{frame}

\begin{frame}\frametitle{Our first Fortran program}
  \begin{itemize}
    \item We first consider compiling a simple hello world program, sources given below.
      \lstinputlisting[language=Fortran]{../../examples/crash/hello.f}
  \end{itemize}
\end{frame}

\begin{frame}\frametitle{Our first Fortran program contd.}
  \begin{itemize}
    \item We only have a single source file.
    \item We do not want to be bothered by the arcane formatting rules of Fortran, thus -ffree-form is required.
    \item Alternatively, we can call the file hello.f90 to compile with Fortran90 standards.
    \item We can then do the compilation and linking in a single command. \\
      gfortran -ffree-form -o hello hello.f
    \item We might want to turn on compiler optimizations. \\
      gfortran -ffree-form -o hello hello.f -O2
    \item We might want to include debugging info. \\
      gfortran -ffree-form -o hello hello.f -g
  \end{itemize}
\end{frame}

\begin{frame}\frametitle{Our second C program}
  \begin{itemize}
    \item It is impractical and unadvisable to have all sources of a complex program in a single source file.
    \item We here consider a more involved hello world program.
    \item We put the printing in a separate source file.
      \lstinputlisting[language=C]{../../examples/crash/utils.c}
  \end{itemize}
\end{frame}

\begin{frame}\frametitle{Our second C program contd.}
  \begin{itemize}
    \item We have to write a \emph{header file} so we can use these functions in another source.
      \lstinputlisting[language=C]{../../examples/crash/utils.h}
  \end{itemize}
\end{frame}

\begin{frame}\frametitle{Our second C program contd.}
  \begin{itemize}
    \item Finally we have the main program
      \lstinputlisting[language=C,basicstyle=\tiny]{../../examples/crash/hello2.c}
  \end{itemize}
\end{frame}

\begin{frame}\frametitle{Our second C program contd.}
  \begin{itemize}
    \item In theory we can compile in one step as we did for the first program. \\
      gcc -o hello2 hello2.c utils.c
    \item Problem with this approach is that all sources are recompiled every time, even
          if nothing changed in some of them.
    \item Better approach: First sources are compiled into objects, then objects are linked together. \\
      gcc -c -o hello2.o hello2.c \\
      gcc -c -o utils.o utils.c \\
      gcc -o hello2 hello2.o utils.o
    \item Same approach for Fortran programs.
  \end{itemize}
\end{frame}

\begin{frame}\frametitle{Our third C program}
  \begin{itemize}
    \item Often we want to share sources between several programs.
    \item The most convenient way to do so is to link objects into a library, and then reuse the library.
    \item We here consider an even more involved hello world program.
    \item We put the printing in a separate source file (plus a header).
      \lstinputlisting[language=C]{../../examples/crash/utils.c}
  \end{itemize}
\end{frame}

\begin{frame}\frametitle{Our third C program contd.}
  \begin{itemize}
    \item We put another printing in another source file (plus a header).
      \lstinputlisting[language=C]{../../examples/crash/utils2.c}
  \end{itemize}
\end{frame}

\begin{frame}\frametitle{Our third C program contd.}
  \begin{itemize}
    \item A main program
      \lstinputlisting[language=C,basicstyle=\tiny]{../../examples/crash/hello3.c}
  \end{itemize}
\end{frame}

\begin{frame}\frametitle{Our third C program contd.}
  \begin{itemize}
    \item We link the printing functions into a library. \\
      gcc -c -o utils.o utils.c \\
      gcc -c -o utils2.o utils2.c \\
      ar r libutils.a utils.o utils2.o
    \item Finally we build the main program. \\
      gcc -c -o hello3.o hello3.c \\
      gcc -o hello3 hello3.o libutils.a 
  \end{itemize}
\end{frame}

\begin{frame}\frametitle{Our third C program contd.}
  \begin{itemize}
    \item Another main program
      \lstinputlisting[language=C,basicstyle=\tiny]{../../examples/crash/hello4.c}
  \end{itemize}
\end{frame}

\begin{frame}\frametitle{Our third C program contd.}
  \begin{itemize}
    \item We can then link second main program against the same library. \\
      gcc -c -o hello4.o hello4.c \\
      gcc -o hello4 hello4.o libutils.a 
    \item A more common way to link against libraries; \\
      gcc -o hello4 hello4.o -L. -lutils
  \end{itemize}
\end{frame}

\begin{frame}\frametitle{Automating builds}
  \begin{itemize}
    \item Typing all these compilation, archiving and linking commands is tedious and error prone.
    \item There are tools to handle this - a common choice is the (GNU) \emph{make} system.
    \item We have to write a Makefile with instructions. These are text files with a particular syntax.
  \end{itemize}
\end{frame}

\begin{frame}\frametitle{Automating builds contd.}
  \lstinputlisting[language=make,basicstyle=\tiny]{../../examples/crash/Makefile}
\end{frame}

\begin{frame}\frametitle{However..}
  \begin{itemize}
    \item Make does not take care of build dependencies, that is locating required libraries etc.
    \item Make does not take care of setting up compiler flags. These are often machine specific, thus 
          hardcoding them is not a good idea.
    \item There exists other tools to handle this.
    \item Traditionally: GNU autotools has been used a lot.
    \item Very powerful, very confusing, steep learning curve.
  \end{itemize}
\end{frame}

\begin{frame}\frametitle{CMake}
  \begin{itemize}
    \item CMake is easier to grasp than autotools.
    \item The C stands for \emph{cross platform}, so the idea is that
          you generate build systems for the different platform based on a common script.
    \item CMake can generate Makefiles for unix, xcode projects for OSX and 
           .vcxproj files for visual studio.
    \item Simplifies cross platform development since you do not have to manually maintain build systems.
  \end{itemize}
\end{frame}

\begin{frame}\frametitle{Anatomy of a CMake script}
\lstinputlisting{test/CMakeLists.txt}
\end{frame}

\begin{frame}\frametitle{Handling extra compiler flags}
\lstinputlisting{test/CMakeLists-openmp.txt}
\end{frame}

\begin{frame}\frametitle{Optional flags}
\lstinputlisting{test/CMakeLists-openmpopt.txt}
\end{frame}

\begin{frame}\frametitle{Out of tree builds}
\begin{itemize}
  \item Often, it is useful to have several configurations of a program easily available.
  \item Examples of this can be with and without optimizations, with and without OpenMP etc.
  \item CMake makes this very easy, as you can do out-of-tree builds.
  \item This is achieved by
    \lstinputlisting{test/oot.sh}
\end{itemize}
\end{frame}

\begin{frame}\frametitle{Shell scripting}
  \begin{itemize}
    \item A common task when developing/using scientific software is analyzing some
      output from a program, checking for correctness, performance etc.
    \item Modifying the sources for these tasks are often tedious and error prone.
    \item In fact, it can even be impossible if you are using software you do not
      have the sources for.
    \item Shell scripts are very useful for such tasks.
    \item Instead of modifying the program, we write a script which repeatedly
      executes the program and captures the interesting output.
  \end{itemize}
\end{frame}

\begin{frame}\frametitle{Shell scripting contd.}
  \begin{itemize}
    \item We here consider a simple little program which, given a number of repetitions
      and a vector length performs the dotproduct of random vectors.
    \item The program prints the elapsed time to stdout.
    \item We want to analyze how the algorithm scales by repeatedly calling
      the program and catching the given runtime.
  \end{itemize}
\end{frame}

\begin{frame}\frametitle{Shell scripting contd.}
  \lstinputlisting[language=C,basicstyle=\Tiny]{../../examples/crash/timing.c}
\end{frame}

\begin{frame}\frametitle{Shell scripting contd.}
  \lstinputlisting[language=bash,basicstyle=\tiny]{../../examples/crash/approxflop}
  \begin{itemize}
    \item The results.asc file can be readily loaded into Matlab/Octave for analysis.
  \end{itemize}
\end{frame}

\begin{frame}\frametitle{The full hello suite}
\lstinputlisting{../../examples/crash/CMakeLists.txt}
\end{frame}

\end{document}
