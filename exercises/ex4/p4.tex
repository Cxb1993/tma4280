\documentclass[11pt]{article}
\bibliographystyle{plain}
\usepackage{geometry} % see geometry.pdf on how to lay out the page. There's lots.
\usepackage{amsmath,amssymb} 
\usepackage{epsfig,epsf,subfigure}
\geometry{a4paper} 


\begin{document}
 
\LARGE
\begin{center}
TMA4280: Introduction to Supercomputing
\end{center}
\vspace{1in}

\begin{center}
{\bf Problem set 4}
\end{center}

\Large
\vspace{0.5in}
\begin{center}
Spring 2014
\end{center}

\vspace{0.5in}

{\bf NOTE THAT}:
\begin{itemize}
  \item this problem set is mandatory;
  \item you can work on this problem in groups with up to 3 members;
  \item a report describing the solution should be written (as continous text);
  \item the source code should be handed in with the report (do not attach it to the report);
  \item please make sure that you have answered all the questions; 
  \item the due date is Monday, February 17, 2014;
  \item the report will count 10\% towards the final grade.
\end{itemize}

\vspace{0.5in}

\begin{center}
\copyright Einar M. R{\o}nquist \\
Department of Mathematical Sciences\\
NTNU, N-7491 Trondheim, Norway\\
All rights reserved
\end{center}

\large

\newpage

\noindent 
Consider a vector $\underline{v} \in \mathbb{R}^n$ where the vector elements are defined as
\begin{align}
  v(i) = \frac{1}{i^2}, \qquad i=1,\ldots, n. 
\end{align}
We want to compute the sum of all the vector elements numerically, i.e., 
we want to find $S_n$ where
\begin{align}
  S_n = \sum_{i=1}^n v(i).
\end{align}
Note that the limit $S = \lim_{n\rightarrow\infty} S_n = \pi^2/6$; e.g., see Rottman.
\vspace{1cm}

\begin{enumerate}
\item Write a program (either in C or in FORTRAN) which will do the following: \\
- generate the vector $\underline{v}$; \\
- compute the sum $S_n$ in double precision on a single processor; \\
- compute the difference $S-S_n$ for $n=2^k$, with $k=3,\ldots,14$; \\
- print out the difference $S-S_n$ in double precision.

\item Do the necessary changes to utilize shared memory parallelization
through OpenMP.

\item Write a program to compute the sum $S_n$ using $P$ processors
where $P$ is a power of 2, and a distributed memory model (MPI).
The program should work as follows:
Only processor 0 should be responsible for generating the vector elements. 
Processor 0 should partition and distribute the vector elements evenly 
among all the processors. 
Each processor should be responsible for summing up its own part. 
At the end, all the partial sums should be added together and made available 
on processor 0 for printout. Report the difference $S-S_n$ in double precision 
for different values of $n$.

\item Confirm that your program also works when you are using OpenMP/MPI in combination.

\item Which MPI calls are convenient/necessary to use?
\item Compare the difference $S-S_n$ from the single-processor program and the 
multi-processor program when $P=2$ and when $P=8$. 
Should the answer be the same in all these cases?
\item Compare the memory requirement per processor for the 
single-processor program and the multi-processor program when $n\gg1 $.
\item How many floating point operations are needed to generate the vector $\underline{v}$?\\
How many floating point operations are needed to compute $S_n$ in (2)? \\  
Is the multi-processor program load-balanced?
\item Do you think parallel processing is attractive to use for solving this problem?
%\item Is there an analytic answer for $S_n$ when $n \rightarrow \infty$?
\end{enumerate}


\end{document}

Initially, the other processes do not know the content of the vector $\underline{v}$
or what $n$ is. The only thing known to the other processes are: 
(i) the vector length $n$ is a power of 2; 
(ii) the number of processes is a power of 2; 
(iii) the vector $\underline{v}$ will be split into $P$ equal parts; 
(iv) each process is responsible for summing up its own part.  
At the end, all the partial sums are added together on process 0
and printed out. 


\subsection*{Exercise 3}

Let $\underline{Q}$ be an $(n-1)\times (n-1)$ matrix 
defined by the matrix elements
\begin{eqnarray*}
Q_{ij} = \sqrt{\frac{2}{n}} \cdot sin(\frac{i\,j\,\pi}{n}) \,\,\, ,
\hspace{.5in}1\,\,\leq\,\, i,j\,\,\leq\,\, n-1 \,\,\, .
\end{eqnarray*}

\begin{enumerate}
\item Make a program (either in C or in FORTRAN) which computes 
the matrix-matrix product $\underline{M} = \underline{Q}^T
\underline{Q}$ for a general $n$. 
\item Test the correctness of your program by experimenting
with different values of parameter $n$. \\
Hint: does the matrix $\underline{M}$ have any characteristic 
structure?
\end{enumerate}
