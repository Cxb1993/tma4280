\documentclass[11pt]{article}
\bibliographystyle{plain}
\usepackage{geometry} % see geometry.pdf on how to lay out the page. There's lots.
\usepackage{amsmath,amssymb} 
\usepackage{epsfig,epsf,subfigure}
\geometry{a4paper} 


%\documentclass{article}

\setlength{\parindent}{0pt}
\setlength{\parskip}{1ex plus 0.5ex minus 0.2ex}

\begin{document}
 
\LARGE
\begin{center}
TMA4280: Introduction to Supercomputing
\end{center}
\vspace{1in}

\begin{center}
{\bf Suggested solutions} \\
Problem set 3
\end{center}

\Large
\vspace{0.5in}
\begin{center}
Spring 2012
\end{center}

\vspace{0.5in}

\begin{center}
\copyright Einar M. R{\o}nquist \\
Department of Mathematical Sciences\\
NTNU, N-7491 Trondheim, Norway\\
All rights reserved
\end{center}

\large

\newpage

\subsection*{Exercise 1}
Each student is advised to check this out for him- or her-self. \\

Read about the LINPACK benchmark at (for example):\\
\texttt{http://www.top500.org/project/linpack}

\subsection*{Exercise 2}
\begin{enumerate}
\item A bus-based system is an example of a shared-memory system.
All the processors can access any physical memory location 
in the system. All memory locations are ``equidistant'' to all processors 
in the sense of having the same memory access time. 
This type of hardware configuration 
is an example of a Symmetric MultiProcessor (SMP). 

The limitation of such a configuration lies in the fact that 
the total bandwidth is fixed (and given by the bus). 
We can easily add more processors to the bus at a low cost, 
but all the processors must share the same bus. 
This limits the scalability of a bus-based approach.

Note that the use of caches can reduce the bandwidth demand.
As long as the needed data can be found in cache, there is 
no need to use the bus. However, the data stored in caches
are replicated from main memory, and a system for keeping the 
caches ``consistent'' (or updated) can become a challenge.

\item Similar to a bus-based system, a crossbar is also 
associated with a shared-memory system. It also represent an
example of a  Symmetric MultiProcessor (SMP) system. 
Unlike a bus-based system, there are multiple paths 
available between a processor and a memory unit,
and between an I/O unit and a memory unit.
By expanding the number of pathways (or the size of the crossbar)
as the number of processors increases, a much more scalable 
system can be constructed. However, the price for doing this 
is high, in particular, for larger systems. Hence, a crossbar 
is typically only used for shared-memory systems up to 
a certain number of processors (e.g., 16 or 64).

\item A mesh topology is associated with a distributed memory 
computer, i.e., a message-passing architecture.
Each processor is here connected to a few neighboring processors
in a very regular pattern. The connections (or network topology)
resemble a structured 2d mesh or a strucured 3d mesh. 
In a 2d mesh, each processor is typically connected to 4 
neighboring processors (East, West, North and South),
except for the processors corresponding to the ``boundary'' 
of the mesh. A variant of the mesh topology is the toroid
with special connections between the processors along the 
``boundary'' of the mesh.

\item In a shared-memory architecture, all the processors
have access to all the available memory, i.e., all the 
processors have direct access to the global address space.
In a distributed memory system, each processor has only 
local memory access; data stored in memory modules associated 
with other processors can only be reached via explicit 
message-passing (e.g., send and receive commands).

\item Yes. 
\end{enumerate}

\subsection*{Exercise 3}

\begin{enumerate}
\item Each character uses a single byte of memory. 
Hence, we need 80 bytes to send 80 characters.
\item Each integer uses four bytes of memory (or 32 bits). 
Hence, we need 4KB to send 1K of integers. 
\item Each floating point number in double precision uses eight bytes of memory 
(or 64 bits). Hence, we need 8KB to send 1K of double precision numbers. 
\end{enumerate}

\subsection*{Exercise 4}
No. For example, we can receive a message using MPI\_ANY\_TAG. 

\end{document}
